% Options for packages loaded elsewhere
\PassOptionsToPackage{unicode}{hyperref}
\PassOptionsToPackage{hyphens}{url}
%
\documentclass[
  ignorenonframetext,
  aspectratio=169,
  russian,
]{beamer}
\usepackage{pgfpages}
\setbeamertemplate{caption}[numbered]
\setbeamertemplate{caption label separator}{: }
\setbeamercolor{caption name}{fg=normal text.fg}
\beamertemplatenavigationsymbolshorizontal
% Prevent slide breaks in the middle of a paragraph
\widowpenalties 1 10000
\raggedbottom
\setbeamertemplate{part page}{
  \centering
  \begin{beamercolorbox}[sep=16pt,center]{part title}
    \usebeamerfont{part title}\insertpart\par
  \end{beamercolorbox}
}
\setbeamertemplate{section page}{
  \centering
  \begin{beamercolorbox}[sep=12pt,center]{section title}
    \usebeamerfont{section title}\insertsection\par
  \end{beamercolorbox}
}
\setbeamertemplate{subsection page}{
  \centering
  \begin{beamercolorbox}[sep=8pt,center]{subsection title}
    \usebeamerfont{subsection title}\insertsubsection\par
  \end{beamercolorbox}
}
\AtBeginPart{
  \frame{\partpage}
}
\AtBeginSection{
  \ifbibliography
  \else
    \frame{\sectionpage}
  \fi
}
\AtBeginSubsection{
  \frame{\subsectionpage}
}

\usepackage{amsmath,amssymb}
\usepackage{iftex}
\ifPDFTeX
  \usepackage[T1]{fontenc}
  \usepackage[utf8]{inputenc}
  \usepackage{textcomp} % provide euro and other symbols
\else % if luatex or xetex
  \usepackage{unicode-math}
  \defaultfontfeatures{Scale=MatchLowercase}
  \defaultfontfeatures[\rmfamily]{Ligatures=TeX,Scale=1}
\fi
\usepackage{lmodern}
\ifPDFTeX\else  
    % xetex/luatex font selection
\fi
% Use upquote if available, for straight quotes in verbatim environments
\IfFileExists{upquote.sty}{\usepackage{upquote}}{}
\IfFileExists{microtype.sty}{% use microtype if available
  \usepackage[]{microtype}
  \UseMicrotypeSet[protrusion]{basicmath} % disable protrusion for tt fonts
}{}
\usepackage{xcolor}
\newif\ifbibliography
\setlength{\emergencystretch}{3em} % prevent overfull lines
\setcounter{secnumdepth}{-\maxdimen} % remove section numbering


\providecommand{\tightlist}{%
  \setlength{\itemsep}{0pt}\setlength{\parskip}{0pt}}\usepackage{longtable,booktabs,array}
\usepackage{calc} % for calculating minipage widths
\usepackage{caption}
% Make caption package work with longtable
\makeatletter
\def\fnum@table{\tablename~\thetable}
\makeatother
\usepackage{graphicx}
\makeatletter
\newsavebox\pandoc@box
\newcommand*\pandocbounded[1]{% scales image to fit in text height/width
  \sbox\pandoc@box{#1}%
  \Gscale@div\@tempa{\textheight}{\dimexpr\ht\pandoc@box+\dp\pandoc@box\relax}%
  \Gscale@div\@tempb{\linewidth}{\wd\pandoc@box}%
  \ifdim\@tempb\p@<\@tempa\p@\let\@tempa\@tempb\fi% select the smaller of both
  \ifdim\@tempa\p@<\p@\scalebox{\@tempa}{\usebox\pandoc@box}%
  \else\usebox{\pandoc@box}%
  \fi%
}
% Set default figure placement to htbp
\def\fps@figure{htbp}
\makeatother

\IfFileExists{plex-otf.sty}{
  %% Full TeXlive
  % \usepackage[%
  %   % math,
  %   RM={Scale=0.94},SS={Scale=0.94},SScon={Scale=0.94},TT={Scale=MatchLowercase,FakeStretch=0.9},DefaultFeatures={Ligatures=Common}
  % ]{plex-otf}
}{
  %% TinyTeX
  \usepackage{libertine}
}

%%% Load theme
% https://deic.uab.cat/~iblanes/beamer_gallery/
\IfFileExists{beamerthemegotham.sty}{
  %% Full TeXlive
  \usetheme{gotham}
  \gothamset{
    numbering=totalpagenumber,
    parttocframe default=off,
    sectiontocframe default=off,
    subsectiontocframe default=off,
  }
}{
  %% TinyTeX
  \usetheme{Madrid}
}
\makeatletter
\@ifpackageloaded{caption}{}{\usepackage{caption}}
\AtBeginDocument{%
\ifdefined\contentsname
  \renewcommand*\contentsname{Содержание}
\else
  \newcommand\contentsname{Содержание}
\fi
\ifdefined\listfigurename
  \renewcommand*\listfigurename{Список иллюстраций}
\else
  \newcommand\listfigurename{Список иллюстраций}
\fi
\ifdefined\listtablename
  \renewcommand*\listtablename{Список таблиц}
\else
  \newcommand\listtablename{Список таблиц}
\fi
\ifdefined\figurename
  \renewcommand*\figurename{Рисунок}
\else
  \newcommand\figurename{Рисунок}
\fi
\ifdefined\tablename
  \renewcommand*\tablename{Таблица}
\else
  \newcommand\tablename{Таблица}
\fi
}
\@ifpackageloaded{float}{}{\usepackage{float}}
\floatstyle{ruled}
\@ifundefined{c@chapter}{\newfloat{codelisting}{h}{lop}}{\newfloat{codelisting}{h}{lop}[chapter]}
\floatname{codelisting}{Список}
\newcommand*\listoflistings{\listof{codelisting}{Листинги}}
\makeatother
\makeatletter
\makeatother
\makeatletter
\@ifpackageloaded{caption}{}{\usepackage{caption}}
\@ifpackageloaded{subcaption}{}{\usepackage{subcaption}}
\makeatother

\ifLuaTeX
\usepackage[bidi=basic]{babel}
\else
\usepackage[bidi=default]{babel}
\fi
\babelprovide[main,import]{russian}
\babelprovide[import]{english}
% get rid of language-specific shorthands (see #6817):
\let\LanguageShortHands\languageshorthands
\def\languageshorthands#1{}
\usepackage{csquotes}
\usepackage{bookmark}

\IfFileExists{xurl.sty}{\usepackage{xurl}}{} % add URL line breaks if available
\urlstyle{same} % disable monospaced font for URLs
\hypersetup{
  pdftitle={Лабораторная работа №1},
  pdfauthor={София Черная},
  pdflang={ru-RU},
  hidelinks,
  pdfcreator={LaTeX via pandoc}}


\title{Лабораторная работа №1}
\subtitle{Модель экспоненциального роста}
\author{София Черная}
\date{2026-02-21}

\begin{document}
\frame{\titlepage}

\renewcommand*\contentsname{Содержание}
\begin{frame}[allowframebreaks]
  \frametitle{Содержание}
  \setcounter{tocdepth}{1}
  \tableofcontents
\end{frame}

\section{Информация}\label{ux438ux43dux444ux43eux440ux43cux430ux446ux438ux44f}

\begin{frame}{Докладчик}
\phantomsection\label{ux434ux43eux43aux43bux430ux434ux447ux438ux43a}
\begin{columns}[c]
\begin{column}{0.7\linewidth}
\begin{itemize}[<+->]
\tightlist
\item
  София Черная
\item
  Студентка
\item
  Российский университет дружбы народов
\item
  \href{mailto:1132236043@pfur.ru}{\nolinkurl{1132236043@pfur.ru}}
\end{itemize}
\end{column}

\begin{column}{0.3\linewidth}
\pandocbounded{\includegraphics[keepaspectratio]{image/me.jpeg}}
\end{column}
\end{columns}
\end{frame}

\section{Цель и
задачи}\label{ux446ux435ux43bux44c-ux438-ux437ux430ux434ux430ux447ux438}

\begin{frame}{Цель работы}
\phantomsection\label{ux446ux435ux43bux44c-ux440ux430ux431ux43eux442ux44b}
Изучение модели экспоненциального роста и освоение методов имитационного
моделирования с использованием языка Julia и пакета DrWatson.
\end{frame}

\begin{frame}{Задачи}
\phantomsection\label{ux437ux430ux434ux430ux447ux438}
\begin{itemize}[<+->]
\tightlist
\item
  Создать рабочее пространство для курса
\item
  Настроить Git и GitHub
\item
  Реализовать модель экспоненциального роста в базовом варианте
\item
  Провести параметрическое исследование модели
\item
  Оформить отчёт с использованием литературного программирования
\end{itemize}
\end{frame}

\section{Теоретическая
часть}\label{ux442ux435ux43eux440ux435ux442ux438ux447ux435ux441ux43aux430ux44f-ux447ux430ux441ux442ux44c}

\begin{frame}{Модель экспоненциального роста}
\phantomsection\label{ux43cux43eux434ux435ux43bux44c-ux44dux43aux441ux43fux43eux43dux435ux43dux446ux438ux430ux43bux44cux43dux43eux433ux43e-ux440ux43eux441ux442ux430}
\begin{itemize}[<+->]
\item
  Описывается дифференциальным уравнением:

  \[ \frac{du}{dt} = \alpha u \]
\item
  Решение: \(u(t) = u_0 e^{\alpha t}\)
\item
  Характеристики:

  \begin{itemize}[<+->]
  \tightlist
  \item
    Скорость роста пропорциональна текущему значению
  \item
    Время удвоения: \(T_2 = \ln(2)/\alpha\)
  \item
    Постоянно во времени
  \end{itemize}
\end{itemize}
\end{frame}

\section{Выполнение
работы}\label{ux432ux44bux43fux43eux43bux43dux435ux43dux438ux435-ux440ux430ux431ux43eux442ux44b}

\begin{frame}{Настройка окружения}
\phantomsection\label{ux43dux430ux441ux442ux440ux43eux439ux43aux430-ux43eux43aux440ux443ux436ux435ux43dux438ux44f}
\begin{itemize}[<+->]
\tightlist
\item
  Установлен Git и выполнена базовая настройка (рис. 1-3)
\item
  Сгенерирован SSH-ключ для GitHub (рис. 4-5)
\item
  Установлены Node.js, npm, pnpm и git-flow (рис. 6-11)
\end{itemize}

\begin{figure}

{\centering \includegraphics[width=0.5\linewidth,height=\textheight,keepaspectratio]{image/1.png}

}

\caption{Установка Git}

\end{figure}%
\end{frame}

\begin{frame}[fragile]{Создание репозитория}
\phantomsection\label{ux441ux43eux437ux434ux430ux43dux438ux435-ux440ux435ux43fux43eux437ux438ux442ux43eux440ux438ux44f}
\begin{itemize}[<+->]
\tightlist
\item
  Репозиторий создан из шаблона
  \texttt{course-directory-student-template} (рис. 12-14)
\item
  Клонирован на локальную машину (рис. 15)
\item
  Выполнена инициализация курса (рис. 16)
\item
  Изменения отправлены на GitHub (рис. 17-18)
\end{itemize}

\begin{figure}

{\centering \includegraphics[width=0.5\linewidth,height=\textheight,keepaspectratio]{image/13.png}

}

\caption{Создание репозитория}

\end{figure}%
\end{frame}

\begin{frame}[fragile]{Настройка Git Flow}
\phantomsection\label{ux43dux430ux441ux442ux440ux43eux439ux43aux430-git-flow}
\begin{itemize}[<+->]
\tightlist
\item
  Инициализирован Git Flow с префиксом тегов \texttt{v} (рис. 19)
\item
  Созданы ветки master и develop (рис. 20)
\item
  Создан релиз версии 1.0.0 (рис. 21)
\item
  Установлен \texttt{standard-changelog} (рис. 22)
\item
  Создан CHANGELOG.md (рис. 23-25)
\end{itemize}

\begin{figure}

{\centering \includegraphics[width=0.5\linewidth,height=\textheight,keepaspectratio]{image/19.png}

}

\caption{Инициализация Git Flow}

\end{figure}%
\end{frame}

\begin{frame}[fragile]{Установка Julia и DrWatson}
\phantomsection\label{ux443ux441ux442ux430ux43dux43eux432ux43aux430-julia-ux438-drwatson}
\begin{itemize}[<+->]
\tightlist
\item
  Установлена Julia через snap (рис. 26-27)
\item
  Установлен пакет DrWatson (рис. 28)
\item
  Создан проект \texttt{project} в папке \texttt{labs/lab01/} (рис. 29)
\item
  Установлены необходимые пакеты (рис. 30)
\item
  Создан тестовый скрипт (рис. 31-33)
\end{itemize}

\begin{figure}

{\centering \includegraphics[width=0.5\linewidth,height=\textheight,keepaspectratio]{image/28.png}

}

\caption{Установка DrWatson}

\end{figure}%
\end{frame}

\section{Реализация
модели}\label{ux440ux435ux430ux43bux438ux437ux430ux446ux438ux44f-ux43cux43eux434ux435ux43bux438}

\begin{frame}[fragile]{Базовый вариант}
\phantomsection\label{ux431ux430ux437ux43eux432ux44bux439-ux432ux430ux440ux438ux430ux43dux442}
\begin{itemize}[<+->]
\tightlist
\item
  Создан скрипт \texttt{01\_exponential\_growth.jl} (рис. 34-35)
\item
  Реализовано решение уравнения \(du/dt = \alpha u\)
\item
  Параметры: \(u_0 = 1.0\), \(\alpha = 0.3\), \(t \in [0, 10]\)
\end{itemize}

\begin{figure}

{\centering \includegraphics[width=0.5\linewidth,height=\textheight,keepaspectratio]{image/35.png}

}

\caption{Код базовой модели}

\end{figure}%
\end{frame}

\begin{frame}{Результаты базового эксперимента}
\phantomsection\label{ux440ux435ux437ux443ux43bux44cux442ux430ux442ux44b-ux431ux430ux437ux43eux432ux43eux433ux43e-ux44dux43aux441ux43fux435ux440ux438ux43cux435ux43dux442ux430}
\begin{itemize}[<+->]
\tightlist
\item
  Получены первые 5 строк результатов (рис. 36)
\item
  Аналитическое время удвоения: 2.31
\item
  Построен график экспоненциального роста (рис. 37)
\end{itemize}

\begin{figure}

{\centering \includegraphics[width=0.6\linewidth,height=\textheight,keepaspectratio]{image/37.png}

}

\caption{График экспоненциального роста}

\end{figure}%
\end{frame}

\begin{frame}{Литературный стиль}
\phantomsection\label{ux43bux438ux442ux435ux440ux430ux442ux443ux440ux43dux44bux439-ux441ux442ux438ux43bux44c}
\begin{itemize}[<+->]
\tightlist
\item
  Скрипт преобразован в литературный стиль (рис. 38)
\item
  Добавлены комментарии с описанием модели
\item
  Запуск дал аналогичные результаты (рис. 39-40)
\end{itemize}

\begin{figure}

{\centering \includegraphics[width=0.5\linewidth,height=\textheight,keepaspectratio]{image/38.png}

}

\caption{Код в литературном стиле}

\end{figure}%
\end{frame}

\begin{frame}[fragile]{Генерация производных форматов}
\phantomsection\label{ux433ux435ux43dux435ux440ux430ux446ux438ux44f-ux43fux440ux43eux438ux437ux432ux43eux434ux43dux44bux445-ux444ux43eux440ux43cux430ux442ux43eux432}
\begin{itemize}[<+->]
\tightlist
\item
  Создан скрипт \texttt{tangle.jl} (рис. 41-42)
\item
  Сгенерированы:

  \begin{itemize}[<+->]
  \tightlist
  \item
    Чистый код
  \item
    Jupyter Notebook
  \item
    Quarto-документация (рис. 43)
  \end{itemize}
\end{itemize}
\end{frame}

\section{Параметрическое
исследование}\label{ux43fux430ux440ux430ux43cux435ux442ux440ux438ux447ux435ux441ux43aux43eux435-ux438ux441ux441ux43bux435ux434ux43eux432ux430ux43dux438ux435}

\begin{frame}[fragile]{Скрипт для параметрического исследования}
\phantomsection\label{ux441ux43aux440ux438ux43fux442-ux434ux43bux44f-ux43fux430ux440ux430ux43cux435ux442ux440ux438ux447ux435ux441ux43aux43eux433ux43e-ux438ux441ux441ux43bux435ux434ux43eux432ux430ux43dux438ux44f}
\begin{itemize}[<+->]
\tightlist
\item
  Создан скрипт \texttt{02\_exponential\_growth.jl} (рис. 44)
\item
  Добавлена поддержка Julia в \texttt{\_quarto.yml} (рис. 45)
\item
  Исследуемые значения \(\alpha = 0.1, 0.3, 0.5, 0.8, 1.0\) (рис. 46)
\end{itemize}

\begin{figure}

{\centering \includegraphics[width=0.5\linewidth,height=\textheight,keepaspectratio]{image/44.png}

}

\caption{Код параметрического исследования}

\end{figure}%
\end{frame}

\begin{frame}[fragile]{Результаты сканирования}
\phantomsection\label{ux440ux435ux437ux443ux43bux44cux442ux430ux442ux44b-ux441ux43aux430ux43dux438ux440ux43eux432ux430ux43dux438ux44f}
\begin{itemize}[<+->]
\tightlist
\item
  Процесс параметрического сканирования (рис. 47)
\item
  Сводная таблица результатов (рис. 48)
\item
  Генерация форматов через \texttt{tangle.jl} (рис. 49)
\end{itemize}

\begin{figure}

{\centering \includegraphics[width=0.6\linewidth,height=\textheight,keepaspectratio]{image/48.png}

}

\caption{Сводная таблица результатов}

\end{figure}%
\end{frame}

\section{Визуализация
результатов}\label{ux432ux438ux437ux443ux430ux43bux438ux437ux430ux446ux438ux44f-ux440ux435ux437ux443ux43bux44cux442ux430ux442ux43eux432}

\begin{frame}{Сравнение траекторий}
\phantomsection\label{ux441ux440ux430ux432ux43dux435ux43dux438ux435-ux442ux440ux430ux435ux43aux442ux43eux440ux438ux439}
\begin{itemize}[<+->]
\tightlist
\item
  Построены графики для всех значений \(\alpha\) (рис. 57)
\item
  Чем больше \(\alpha\), тем быстрее рост популяции
\end{itemize}

\begin{figure}

{\centering \includegraphics[width=0.6\linewidth,height=\textheight,keepaspectratio]{image/57.png}

}

\caption{Параметрическое исследование}

\end{figure}%
\end{frame}

\begin{frame}{Зависимость времени удвоения}
\phantomsection\label{ux437ux430ux432ux438ux441ux438ux43cux43eux441ux442ux44c-ux432ux440ux435ux43cux435ux43dux438-ux443ux434ux432ux43eux435ux43dux438ux44f}
\begin{itemize}[<+->]
\tightlist
\item
  Построена зависимость \(t_2 = \ln(2)/\alpha\) (рис. 58-59)
\item
  Численные решения совпадают с теоретической кривой
\end{itemize}

\begin{figure}

{\centering \includegraphics[width=0.6\linewidth,height=\textheight,keepaspectratio]{image/58.png}

}

\caption{Зависимость времени удвоения от α}

\end{figure}%
\end{frame}

\begin{frame}{Дополнительные графики}
\phantomsection\label{ux434ux43eux43fux43eux43bux43dux438ux442ux435ux43bux44cux43dux44bux435-ux433ux440ux430ux444ux438ux43aux438}
\begin{itemize}[<+->]
\tightlist
\item
  Зависимость времени вычисления от \(\alpha\) (рис. 50)
\item
  Функция-обёртка в Jupyter (рис. 54-56)
\end{itemize}

\begin{figure}

{\centering \includegraphics[width=0.5\linewidth,height=\textheight,keepaspectratio]{image/50.png}

}

\caption{Зависимость времени вычисления}

\end{figure}%
\end{frame}

\section{Заключение}\label{ux437ux430ux43aux43bux44eux447ux435ux43dux438ux435}

\begin{frame}{Выводы}
\phantomsection\label{ux432ux44bux432ux43eux434ux44b}
\begin{itemize}[<+->]
\tightlist
\item
  Создано рабочее пространство для курса
\item
  Настроены Git, GitHub и Git Flow
\item
  Установлены Julia и необходимые пакеты
\item
  Реализована модель экспоненциального роста в базовом и параметрическом
  вариантах
\item
  Код оформлен в литературном стиле
\item
  Проведён анализ зависимости времени удвоения от \(\alpha\)
\item
  Подтверждено совпадение численных результатов с теоретической кривой
  \(t_2 = \ln(2)/\alpha\)
\end{itemize}
\end{frame}

\begin{frame}{Спасибо за внимание!}
\phantomsection\label{ux441ux43fux430ux441ux438ux431ux43e-ux437ux430-ux432ux43dux438ux43cux430ux43dux438ux435}
\end{frame}




\end{document}
